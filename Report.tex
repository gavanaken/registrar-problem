\documentclass[11pt, oneside]{article}   	% use "amsart" instead of "article" for AMSLaTeX format
\usepackage{geometry}
\usepackage[]{algorithm2e}
\geometry{letterpaper}
\usepackage{graphicx}
\usepackage{amssymb}
\usepackage{amsthm}
\usepackage{parskip}

\graphicspath{ {images/} }

\newtheorem*{theorem}{Theorem}
\newtheorem*{lemma}{Lemma}
%SetFonts

\title{CS340 - Registrar Project}
\author{Greg Van Aken, Conor Stuart Roe, Russell Gerhard}
%\date{}							% Activate to display a given date or no date

\begin{document}
\maketitle

\section{Checkpoint 1}
\subsection{Description}
This algorithm works by assigning conflict scores to each pair of classes based on teacher assignments and student preferences and assigning each class to be the sole member of a set, then continuously merging sets of classes (designating them to occupy the same time slot) in a greedy manner, on the basis of lowest total set conflict scores. \\

Call the number of classes as $n$. Initialize a "conflict matrix" $M$, an $n \times n$ 2-D matrix for storing conflict scores, setting all values to 0. Additionally, create a union-find object $S$ and assign each class to its own set within $S$. For each teacher, each pair of classes they teach ($c$, $c^\prime$) definitely cannot occupy the same time slot, so the conflict scores $M[c][c^\prime]$ and $M[c^\prime][c]$ are set to {\it Infinity} (or some sufficiently large number). Then each student's preference list is iterated through, and for each class pair $(c_i, c_j)$ in a given student's preference list, compute the conflict score ((4-i) + (4-j)) for that pair (for example, if $i$ is 0 and $j$ is 1, the student prefers both of those classes highly and their overlap will be a cost of 7) and increment $M[c_i][c_j]$ and $M[c_j][c_i]$ by that conflict score. \\

Create a priority queue $Q$ with a min heap to keep track of the pair of sets with lowest conflict score, and add each cell not along the diagonal to $Q$. Sets are identified with unique group ID numbers, and as new sets are created by merging, they are given incrementally larger group ID numbers. While the number of sets in $S$ is greater than the number of class times, classes still haven't been grouped into sufficiently few time slots, so merging must continue. \\

At each step, find the sets $s_i$, $s_j$ with the minimum conflict score by popping from $Q$. It is a common occurrence that oe or both sets specified by $s_i$ and $s_j$ no longer "exist" - that is, the sets they represent have already been merged with some other set, in which case the pair is disregarded. Otherwise, an attempt is made to merge the sets. If $|s_i| + |s_j|$ is less than the number of classrooms, the sets can be merged without creating too large a time slot, so the sets are merged, otherwise that pair is also simply disregarded. When a merge takes place between $s_i$ and $s_j$ to form set $s_k$, the conflict score of any other set $s_o$ with $s_k$ should be the sum of $s_o$'s conflict scores with $s_i$ and $s_j$, since it now conflicts with all the classes in both, so row and column $j$ are added to row and column $i$, and $i$ is designated as the repsentative row/column of group $s_k$. Row/column $j$ continue to exist to avoid matrix resizing, but they are never again looked at. The $i^{th}$ row and column now represent the set $s_k$, which contains all the classes originally in either set $s_i$ or $s_j$. Join the two sets in $S$ as well.

Each iteration reduces the number of sets in $S$ by one. Once the number of sets in $S$ is equal the number of time slots, each class set can be arbitrarily assigned to a time slot.

\subsection{Pseudocode}
\begin{algorithm}
\SetKwFunction{Main}{Main}
\SetKwProg{Fn}{Function}{}{}
\Fn{\Main{Preferences, numClasses, Teachers}}{
let $n$ = len(numClasses);
initialize $M$ = n x n 2-D array of all 0's;

$M$ = setTeach($M$, $Teachers$);

$M$ = setCost($M$, $Preferences$);

createSets(M);


}
\end{algorithm}

\begin{algorithm}
\SetKwFunction{setTeach}{setTeach}
\SetKwProg{Fn}{Function}{}{}
\Fn{\setTeach{M, Teachers}}{
sort $Teachers$ by teacher id; // Teachers is a list of class-teacher pairs

initialize $last$ = Teachers[0];

\While{$i=1 < len(Teachers)$} {
let $teach$ = Teachers[i] \\
\If{teach[1] == last[1]} {
set $M$[teach[0]][last[0]] to $infinity$; // teacher teaches both classes \\
}
set $last$ = Teachers[i] \\
i++\\
}
\Return{$M$}
}
\end{algorithm}

\begin{algorithm}[H]
\SetKwFunction{setCost}{setCost}
\SetKwProg{Fn}{Function}{}{}
\Fn{\setCost{M, Preferences}}{
\For{student in Preferences} {
// Assume Preferences = [[p1,p2,p3,p4],[p1,p2,p3,p4],...]

\For{i in range(0:4)} {
{
\For{j in range(i+1:4)} {
let $c1$ = student[i]; let $c2$ = student[j];

$M$[$c1$][$c2$] += ((4-i) + (4-j))
}
}
}
}
}
\end{algorithm}

\begin{algorithm}[H]
\SetKwFunction{createSets}{createSets}
\SetKwProg{Fn}{Function}{}{}
\Fn{\createSets{M}}{
    \# M is an $n$ by $n$ matrix \\
    Initialize a union-find object, $S$ \\
    \For{each class}{
         Initialize a disjoint set of size one containing that class, stored in $S$ \\
    }
    Initialize a priority queue $P$ with all values from M in it \\
    \While{number of sets in $S$ $\geq$ the number of time slots}{
        Pop $(S_i, S_j)$ \\
	\If {either $S_i$ or $S_j$ has already been merged} {pass}
        \ElseIf{$|S_i|+|S_j| \leq$ the number of classrooms}{
            merge $S_k = S_i \cup S_j$ \\
            Add row and column $j$ to row and column $i$ \\
            Designate row and column $i$ as the location of $S_k$'s conflict scores \\
            Push all of $S_k$'s conflict scores to $P$ \\
        }
    }
    \For{each disjoint set $s$ in $S$}{
        assign $s$ to an unassigned time slot \\
        \For {each class $c$ in $s$}{
            assign $c$ to an unassigned classroom
        }
    }
    \# write output to a file \\
}
\end{algorithm}

\subsection{Time Analysis}

Call the number of teachers $t$, the number of students $s$, and the number of classes $n$. \\

setTeach contains a sort of teachers, which takes $O(t$log$t)$ time. Then it iterates over teachers performing constant time operations, so taking $O(t)$ time. Overall, setTeach is $O(t$log$t)$. \\

setCost contains a triply nested loop, but the number of loops of the two inner loops is fixed rather than related to any input. Specifically, six innermost loops occur per student. Thus the total time is $O(s)$. \\

createSets is a more complex function. The way it is written, initialization of the Union-Find, which contains one set per class, is $O(n)$. Initializing the priority queue is constant, but pushing all $O(n^2)$ values of $M$ to it is $O(n^2$log$(n^2)) = O(n^2$log$n)$. All sets must be merged by the time the priority queue is empty, so the maximum number of iterations of the \textbf{while} is no more than the maximum number of entries ever created in the priority queue $P$. Since every group merge reduces the number of class groups by 1, at most $n-1$ merges may take place, so at most $2n-1$ groups can ever be created before all classes are in one group. Each conflict score is between two groups, so the total number of possible conflicts to store in $P$ is $O((2n-1)^2) = O(n^2)$. The time of popping from $P$ is $O($log$(n^2)) = O($log$n)$. Checking whether a group has already been merged, checking the group sizes of $S_i$ and $S_j$, and merging sets are all constant time with our data structures. Adding rows and pushing them to $P$ can occur in total as many times as there are possible conflict scores, again $O(n^2)$, so it does not actually increase the time complexity of the \textbf{while}. Overall the most complex operations that occurs for every loop are the pop and push to $P$, which are $O($log$n)$, so the total complexity of the \textbf{while} is $O(n^2$log$n)$. Lastly, the double \textbf{for} iterates over all classes, performing a constant time operation of assignment, so its complexity is $O(n)$. The dominating complexity of createSets is $O(n^2$log$n)$. \\

Each function is performed once during the algorithm's run, so the time complexity is their sum $O(t+s+n^2$log$n)$. Assuming that the number of classes taught by each teacher closely hews to an average, the number of teachers is linear relative to the number of classes - under this assumption, the algorithm is $O(s + n^2$log$n)$. \\

\subsection{Discussion}

We had a relatively good intuition when approaching this problem that we had to somehow compute a cost associated with having classes at the same time. One option we considered was to weight the cost according to how many students preferred a given class. In other words, for every student that lists some class as their first choice, add a cost of 4 to that class. For every student who lists it as their second choice, add a cost of 3, etc. We realized that although this would determine a set of highly preferred classes that may have a number of students wanting to take them (and thus should not overlap), it loses the logic about what each student wants. A counter example to our initial approach is that there are two popular courses but they occur in different majors. There may be a lot of students who prefer those classes, but few students will desire to take both courses. Thus, it would be reasonable for those classes to overlap. This led us to our current algorithm which makes decisions based on the cost to each student of having two classes overlap. The aspects of this problem which made it difficult to determine an algorithm are:
\begin{enumerate}
\item{How do we quantify the cost of two classes overlapping?}
\begin{itemize}
\item{We decided to set it up such that merging courses is a direct function of their collective priorities}
\item{An artifact of this is that merging classes 1 and 4 has the same cost as merging classes 2 and 3}
\end{itemize}
\item{After a minimum cost is found and then a merge takes place, the cost of merging any third class with the new set will have to include the total merge cost of the third class with every class already in the set} 
\begin{itemize}
\item{This fact led us to the idea of merging matrix rows/columns to reflect the additive costs when sets are merged}
\end{itemize}
\item{A given teacher cannot teach more than one class at a given time}
\begin{itemize}
\item{This issue led us to perform the initialization step of setting matrix cells coorresponding to classes taught by the same teacher to Infinity}
\end{itemize}
\item{A given time slot cannot have more classes scheduled than the given number of rooms}
\begin{itemize}
\item{Because of this, we decided to check to see if this condition is broken and if so, skip that merge}
\end{itemize}
\end{enumerate}
This algorithm is {\bf greedy} because each time work is done, it generally follows a simple rule: merge sets with the lowest merge cost (if a merge is valid). This algorithm is not too different from constructing a minimum spanning tree using Kruskal's algorithm. Classes can be thought of as nodes in a graph and merge cost can be thought of as the weight of the edge between nodes. Similar to a minimum spanning tree, minimum edge weights are selected as the sets are constructed (merged). 

\end{document}


