\documentclass[11pt, oneside]{article}   	% use "amsart" instead of "article" for AMSLaTeX format
\usepackage{geometry}
\usepackage[]{algorithm2e}
\geometry{letterpaper}
\usepackage{graphicx}
\usepackage{amssymb}
\usepackage{amsthm}
\usepackage{parskip}

\graphicspath{ {images/} }

\newtheorem*{theorem}{Theorem}
\newtheorem*{lemma}{Lemma}
%SetFonts

\title{CS340 - Registrar Project}
\author{Greg Van Aken, Conor Stuart Roe, Russell Gerhard}
%\date{}							% Activate to display a given date or no date

\begin{document}
\maketitle

\section{Checkpoint 2}
\subsection{Description}
This algorithm works by assigning conflict scores to each pair of classes based on teacher assignments and student preferences and assigning each class to be the sole member of a set, then continuously merging sets of classes (designating them to occupy the same time slot) in a greedy manner, on the basis of lowest total set conflict scores. \\

Call the number of classes as $n$. Initialize a "conflict matrix" $M$, an $n \times n$ 2-D matrix for storing conflict scores, setting all values to 0. Additionally, create a union-find object $S$ and assign each class to its own set within $S$. For each teacher, each pair of classes they teach ($c$, $c^\prime$) definitely cannot occupy the same time slot, so the conflict scores $M[c][c^\prime]$ is set to {\it Infinity} (or some sufficiently large number). Then each student's preference list is iterated through, and for each class pair $(c, c^\prime)$ in a given student's preference list, compute the conflict score ((4-index($c$)) + (4-index($c^\prime$))) for that pair (for example, if $c$ is in position 0 (top choice) and $c^\prime$ is in position 1 (next choice), the student prefers both of those classes highly and their overlap will be a cost of 7) and increment $M[c_i][c_j]$  by that conflict score. To ensure that we only need to use the upper triangle of the matrix, we will adopt the convention of always indexing such that $i < j$. \\

Create a priority queue $Q$ with a min heap to keep track of the pair of sets with lowest conflict score, and add each cell not along the diagonal to $Q$. While the number of sets in $S$ is greater the number of class times, classes still haven't been grouped into sufficiently few time slots, so merging must continue. At each step, find the sets $s_i$, $s_j$ with the minimum conflict score by popping from $Q$. If $|s_i| + |s_j|$ is less than the number of classrooms, the sets can be merged without creating too large a time slot, so the sets are merged (otherwise that pair is thrown out of the priority queue). When a merge takes place between $s_i$ and $s_j$ to form set $s_k$, the conflict score of any other set $s_o$ with $s_k$ should be the sum of $s_o$'s conflict scores with $s_i$ and $s_j$, since it now conflicts with all the classes in both, so row $j$ is added to row $i$, column $j$ is added to column $i$, and row $j$ and column $j$ are deleted. The $i^{th}$ row and column now represent the set $s_k$, which contains all the classes originally in either set $s_i$ or $s_j$. Join the two sets in $S$ as well.

Each iteration reduces the number of sets in $S$ by one. Once the number of sets in $S$ is equal the number of time slots, each class set can be arbitrarily assigned to a time slot.

\subsection{Pseudocode}
\begin{algorithm}
\SetKwFunction{Main}{Main}
\SetKwProg{Fn}{Function}{}{}
\Fn{\Main{Preferences, numClasses, Teachers}}{
let $n$ = len(numClasses);
initialize $M$ = n x n 2-D array of all 0's;

$M$ = setTeach($M$, $Teachers$);

$M$ = setCost($M$, $Preferences$);

createSets(M);


}
\end{algorithm}

\begin{algorithm}
\SetKwFunction{setTeach}{setTeach}
\SetKwProg{Fn}{Function}{}{}
\Fn{\setTeach{M, Teachers}}{
sort $Teachers$ by teacher id; // Teachers is a list of class-teacher pairs

initialize $prev$ = Teachers[0];

\While{$i=1 < len(Teachers)$} {
let $teach$ = Teachers[i];

\If{teach[1] == prev[1]} {
set $M$[teach[0]][prev[0]] to $infinity$; // teacher teaches both classes
}
set $prev$ = Teachers[i]; i++;
}
\Return{$M$}
}
\end{algorithm}

\begin{algorithm}
\SetKwFunction{setCost}{setCost}
\SetKwProg{Fn}{Function}{}{}
\Fn{\setCost{M, Preferences}}{
\For{student in Preferences} {
// Assume Preferences = [[p1,p2,p3,p4],[p1,p2,p3,p4],...]

\For{i in range(0:4)} {
{
\For{j in range(i+1:4)} {
let $c1$ = student[i]; let $c2$ = student[j];

$M$[$c1$][$c2$] += ((4-i) + (4-j))
}
}
}
}
}
\end{algorithm}

\begin{algorithm}
\SetKwFunction{createSets}{createSets}
\SetKwProg{Fn}{Function}{}{}
\Fn{\createSets{M}}{
    // M is an $n$ by $n$ matrix
    Initialize a union-find object, $S$ \\
    \For{each class}{
         Initialize a disjoint set of size one containing that class, stored in $S$ \\
    }
    Initialize a priority queue $P$ with all values from M in it \\
    \While{numrow(M) $\geq$ the number of time slots}{
        Pop $(S_i, S_j)$ \\
        \If{$|S_i|+|S_j| \leq$ the number of classrooms}{
            Add column $j$ to column $i$ \\
            Add row $j$ to row $i$ \\
            Delete column and row $j$ \\
        }
    }
    \For{each disjoint set $s$ in $S$}{
        assign $s$ to an unassigned time slot
        // write to a file
        \\
        \For {each class $c$ in $s$}{
            assign $c$ to an unassigned classroom
        }
    }
}
\end{algorithm}

\subsection{Time Analysis}
Time Analysis goes here. $O(n^2)$


\subsection{Discussion}

We had a relatively good intuition when approaching this problem that we had to somehow compute a cost associated with having classes at the same time. One option we considered was to weight the cost according to how many students preferred a given class. In other words, for every student that lists some class as their first choice, add a cost of 4 to that class. For every student who lists it as their second choice, add a cost of 3, etc. We realized that although this would determine a set of highly preferred classes that may have a number of students wanting to take them (and thus should not overlap), it loses the logic about what each student wants. A counter example to our initial approach is that there are two popular courses but they occur in different majors. There may be a lot of students who prefer those classes, but few students will desire to take both courses. Thus, it would be reasonable for those classes to overlap. This led us to our current algorithm which makes decisions based on the cost to each student of having two classes overlap. The aspects of this problem which made it difficult to determine an algorithm are:
\begin{enumerate}
\item{How do we quantify the cost of two classes overlapping?}
\begin{itemize}
\item{We decided to set it up such that merging courses is a direct function of their collective priorities}
\item{An artifact of this is that merging classes 1 and 4 has the same cost as merging classes 2 and 3}
\end{itemize}
\item{After a minimum cost is found and then a merge takes place, the cost of merging any third class with the new set will have to include the total merge cost of the third class with every class already in the set} 
\begin{itemize}
\item{This fact led us to the idea of merging matrix rows/columns to reflect the additive costs when sets are merged}
\end{itemize}
\item{A given teacher cannot teach more than one class at a given time}
\begin{itemize}
\item{This issue led us to perform the initialization step of setting matrix cells coorresponding to classes taught by the same teacher to Infinity}
\end{itemize}
\item{A given time slot cannot have more classes scheduled than the given number of rooms}
\begin{itemize}
\item{Because of this, we decided to check to see if this condition is broken and if so, skip that merge}
\end{itemize}
\end{enumerate}
This algorithm is {\bf greedy} because each time work is done, it generally follows a simple rule: merge sets with the lowest merge cost (if a merge is valid). This algorithm is not too different from constructing a minimum spanning tree using Kruskal's algorithm. Classes can be thought of as nodes in a graph and merge cost can be thought of as the weight of the edge between nodes. Similar to a minimum spanning tree, minimum edge weights are selected as the sets are constructed (merged). 

\subsection{Implementation}
We chose to implement this algorithm in Python. Much of the implementation followed directly from our design and pseudocode. We were able to utilize the {\it unionfind} package hosted on {\it https://pypi.org} (licensed to The Python Software Foundation). We needed to modify the implementation slightly to allow for sets to contain more information (class IDs). We also implemented our own min heap to serve as the priority queue. There was one key addition to the algorithm we noticed when attempting to implement. When a set is popped from the priority queue, it's possible that it has already been {\it unioned} with a previously popped set (which would have then been pushed back onto the queue). Thus, we keep track of what has been {\it unioned} and if we pop a set that has been {\it unioned} already, we throw it away and continue with the next set on the queue. This change has been added to our description and pseudocode.

\subsection{Experimental Analysis (WIP)}
After an initial implementation, we have determined that the algorithm works for the set of basic data. However, in porting this algorithm to real data, we have determined that there are a few extra elements to consider. Most notably, we have concluded that our current algorithm will run slow for large datasets and needs to be optimized (note that data to be filled in). 

One idea we have that we think will drastically cut down the complexity of our algorithm depends on information from the full enrollment data. There are courses offered by the same department that will likely not have any overlapping interest by students. An example of this might be a 100-level (intro) course and a 300-level course along the same subject track. There is likely no reason for a student to be taking an intro course at the same time as an upper-level course in the same subject. An analysis of enrollment data should help us elucidate whether there are cases of students taking multiple courses of differing levels in the same department and if so what those cases are an how many of them occur. After that analysis, we should be able to initially union courses that would likely never have concurrent interest from any one student. This will cut down on the $O(n^2)$ components of our algorithm because we won't need to look at every single pair.

Another similar circumstance to analyze is whether or not certain departments have very little upper-level overlap. If somebody takes a 300-level computer science class, they may not be likely to take a 300-level history class. By identifying these trends in the data, we can hopefully find other likely candidates for co-scheduling. Similarly, this will cut down on needing to enumerate every pairwise course conflicts and improve the performance of the algorithm.

We also noticed that since student enrollment is not necessarily a weighted preference, we should weight each class overlap as 1 regardless of where they appear in a student's preference list. 

\end{document}


